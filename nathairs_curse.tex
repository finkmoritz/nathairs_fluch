\documentclass[]{scrartcl}

\usepackage[german]{babel}
\usepackage[utf8]{inputenc}

\usepackage{calligra} % font
\usepackage{hyperref} % URLs, links
\usepackage{lscape} % landscape mode
\usepackage{tabularx} % tables

\usepackage{tabularx} % tables
\usepackage{xargs} % multiple parameters

\newcolumntype{Y}{>{\centering\arraybackslash}X} % same as 'X' but centered

\newcommandx{\dndtextbox}[2][1=\linewidth]{
	{
		\setlength{\fboxsep}{1em}
		\fbox{
			\parbox{#1}{
				#2
			}
		}
	}
}

\newcommandx{\dndspell}[8][1=-, 2=-, 3=-, 4=-, 5=-, 6=-, 7=~, 8=~]{
	\begin{tabularx}{\linewidth}{X}
		{\Large \bfseries{#1}}\\
		~\\
		{\small \textit{#2}}\\
		~\\
		{\footnotesize {\bfseries Zeitaufwand:} #3}\\
		{\footnotesize {\bfseries Reichweite:} #4}\\
		{\footnotesize {\bfseries Komponenten:} #5}\\
		{\footnotesize {\bfseries Wirkungsdauer:} #6}\\
		~\\
		{\scriptsize #7}\\
		{\scriptsize #8}
	\end{tabularx}
}

\newcommandx{\dndmonsterheader}[5][1=-, 2=-, 3=-, 4=-, 5=-]{
	{\Large \bfseries{#1}}\\
	{\small \textit{#2}}\\
	~\\
	{\footnotesize {\bfseries Rüstungsklasse:} #3}\\
	{\footnotesize {\bfseries Trefferpunkte:} #4}\\
	{\footnotesize {\bfseries Bewegungsrate:} #5}\\
	~\\
}

\newcommandx{\dndmonsterattributes}[6][1=-, 2=-, 3=-, 4=-, 5=-, 6=-]{
	\begin{tabularx}{\linewidth}{YYYYYY}
		{\small {\bfseries STR}} &
		{\small {\bfseries GES}} &
		{\small {\bfseries KON}} &
		{\small {\bfseries INT}} &
		{\small {\bfseries WEI}} &
		{\small {\bfseries CHA}}\\
		
		{\small #1} &
		{\small #2} &
		{\small #3} &
		{\small #4} &
		{\small #5} &
		{\small #6}
	\end{tabularx}
	~\\
}

\newcommandx{\dndmonsterinfo}[4][1=-, 2=-, 3=-, 4=-]{
	{\footnotesize {\bfseries Fertigkeiten:} #1}\\
	{\footnotesize {\bfseries Sinne:} #2}\\
	{\footnotesize {\bfseries Sprachen:} #3}\\
	{\footnotesize {\bfseries Herausforderungsgrad:} #4}\\
	~\\
}

\newcommandx{\dndmonsterfeature}[2]{
	{\footnotesize {\bfseries #1:} #2}\\
	~\\
}

\newcommandx{\dndattack}[5]{
	\textit{#1:} #2 auf Treffer, Reichweite #3, #4. \textit{Treffer:} #5
}


\title{Nathair's Fluch}
\author{Moritz Fink}

\begin{document}

\maketitle

\begin{abstract}
\textit{
Diese Dungeons \& Dragons Kampagne für 2-4 Charaktere (1. Stufe) wurde inspiriert von J.K. Rowling und David Ng (vgl. \url{https://popperfont.files.wordpress.com/2016/12/hpddcampaign1.pdf}). Dieses Skript ist dabei ausschließlich für den Spielleiter vorgesehen.\\
}

Als eines Tages ein mysteriöser Brief auf dem Tisch liegt, müssen sich normale Muggel als Zauberer beweisen, um Luna's Zauberstab sicher nach Hogwarts zu bringen. Dort angekommen, fällt diese jedoch in die Hände des bösen Zauberers Nathair und Hogwarts wird zur Ruine. Können sich die mutigen Abenteurer in dieser harten Prüfung beweisen oder ist das das Ende der einst so prunkvollen Zauberschule?
\end{abstract}

\newpage

\tableofcontents

\newpage

\section{Auftakt}

Das Abenteuer beginnt mit einem Brief, welcher die Spieler zunächst in die Kampagne einführt und ihre Charaktere erstellen lässt.

\subsection{Ein mysteriöser Brief}

Die Spieler versammeln sich am Tisch und finden dort Luna's Zauberstab, einen Brief und jeweils einen Teebeutel vor. Der Spielleiter sollte vorerst in keinster Weise eingreifen, damit die Spieler den Brief lesen und entsprechende Aktionen ableiten können. Aus dem Inhalt sollte relativ schnell ersichtlich sein, dass sich die Spieler zuerst Tee machen sollen. Die Zeit bis dieser trinkbar ist sollte mit der Suche nach einem geeigneten Zauberstabersatz und der Charaktererstellung gefüllt werden.

\newpage

{\calligra \large
	
Liebe Muggel,\\

uns bleibt nur wenig Zeit, die Situation zu erklären. Also werden wir uns kurzfassen.\\

Der Zauberstab, welchen ihr besitzt, ist mit äußerster Vorsicht zu behandeln, denn es handelt sich in Wirklichkeit um den Zauberstab unserer Mutter. Große und mächtige Taten wurden damit schon vollführt!\\

Doch als er zu einer Gefahr für die freie Welt der Hexen und Zauberer wurde, haben wir ihn (schlauerweise) als Spielzeug getarnt in einem Muggelgeschäft versteckt, damit er vor den Augen dunkler Mächte verschlossen bleibt. Bedauerlicherweise können wir nicht mehr für eure Sicherheit garantieren, da Nathair’s Schergen herausgefunden haben, wo sich der Zauberstab befindet.\\

Somit tragt ihr nun die Bürde, ihn sicher nach Hogwarts zu bringen und an eine gewisse Mrs. Weasley-Granger zu übergeben.\\

Um euch bei dieser Aufgabe zu helfen, haben wir dem Brief Teebeutel beigelegt. Sucht euch jeweils einen Zauberstabersatz (etwas Zauberstab-artiges) und haltet diesen fest in eurer Hand, während ihr euren Tee genießt. Sobald jeder von euch getrunken hat, fungiert der (echte) Zauberstab als Portschlüssel. Stellt sicher, dass ihr ihn alle gleichzeitig berührt und alles Wichtige mitnehmt!\\

Sobald ihr angekommen seid, steigt \underline{zügig} ein und achtet darauf, dass ihr nicht verfolgt werdet. Findet den Ort an dem \underline{nur ein Stück pro Person} erlaubt ist.\\

Beeilt euch, denn es gilt keine Sekunde zu verlieren!\\

~\\

Hochachtungsvoll\\
Lorcan \& Lysander
}

\newpage

\subsection{Charakterbögen}

Der Spielleiter händigt jeweils einen D\&D Charakterbogen an jeden Spieler aus und führt durch die Charaktererstellung. Diese richtet sich im Großen und Ganzen nach dem D\&D 5e Regelwerk, allerdings mit folgenden Modifikationen:
\begin{itemize}
	\item Jeder Spieler spielt sich selbst
	\item Klasse = Magier
	\item Die Zaubersprüche finden sich in Anhang \ref{appendix-spells}
	\item (optional) Für jeden Spieler füllen die jeweils anderen Spieler den Bogen aus, um dessen Stärken und Schwächen darzustellen
\end{itemize}

\subsection{Du bist ein Zauberer!}

Nun ist alles vorbereitet, um die Spieler ins Abenteuer zu stürzen. Wie im Brief beschrieben, soll nun jeder Spieler seinen Tee trinken während er dessen Zauberstab in der Hand hält. Der Spielleiter kann hier klarmachen, dass bei dieser Prozedur Magie in die Zauberstäbe der Spieler fließt (z.B. "dein Zauberstab strahlt plötzlich Wärme ab").

Anschließend müssen alle Spieler Luna's Zauberstab gleichzeitig berühren, damit dieser sie in seiner Funktion als Portschlüssel zum Gleis $9\frac{3}{4}$ transportiert.

\section{Der Hogwarts Express}

Hier beginnt das klassische D\&D Erlebnis inklusive erster Kämpfe und Rätsel.

\subsection{Gleis $9\frac{3}{4}$}

Nach der Portschlüssel-Reise muss jeder Spieler einen Konstitutions-Rettungswurf bestehen. Schlägt dieser fehl, so übergibt sich der jeweilige Charakter und erleidet 1 Schaden.

Als sich die Charaktere umsehen, wird ihnen klar, dass sie sich auf Gleis $9\frac{3}{4}$ befinden. Der gesamte Bereich ist menschenleer und vor ihnen befindet sich der Hogwarts Express mit {\bfseries sieben} leeren Wägen. Die vorderste Tür (Wagen 1) ist geöffnet und die Lokomotive bläst bereits zur Abfahrt. Die Charaktere befindet sich nach der Ankunft direkt neben dem ersten Wagen.

Vom hintersten Wagen aus nähern sich zwei dunkle Gestalten, welche vergebens versuchen, eine Tür nach der anderen mit Zaubern zu öffnen. Sie scheinen die Charaktere noch nicht bemerkt zu haben.

\dndinfobox[Anzahl der Gegner]{Die Anzahl an Gegner sollte sich hier nach der Anzahl der Charaktere richten. Ein normaler Schwierigkeitsgrad ergibt sich, wenn es genauso viele Gegner wie Charaktere gibt.}

Die Spieler haben nun die Wahl. Entweder steigen sie direkt in den Zug ein (siehe nächster Abschnitt) oder stellen sich den Fremden. Sollten sie sich für letzteres entscheiden, so kommt es relativ schnell zum Kampf, da Nathair's Schergen nicht leicht mit sich verhandeln lassen und erkennen, dass die Abenteurer etwas mit ihrer Mission zu tun haben. Sollte es zu einem Kampf kommen, so gilt das D\&D Regelwerk und die Zaubersprüche in Anhang \ref{appendix-spells}.

Anschließend sollen sich die Charaktere durch die offene Tür in den Zug begeben. Der Spielleiter kann hier die Spieler dazu drängen, indem er mehrfach das Horn zur Abfahrt ertönen lässt.

\subsection{Magische Wagenreihung}

\subsection{Das Gepäckabteil}

\section{Hogwarts}

\subsection{Nathair's Fluch}

\appendix

\begin{landscape}
	\section{\label{appendix-spells}Zauber}
	
	\begin{tabularx}{\linewidth}{XXX}
		\dndspell[Lumos][Zaubertrick der Hervorrufung][Eine Aktion][Selbst][V, S, M (Zauberstab)][Eine Stunde][Während der Wirkungsdauer strahlt dein Zauberstab helles Licht in einem Radius von sechs Metern und dämmriges Licht im Radius von weiteren sechs Metern aus. Du kannst die Farbe des Lichts frei wählen. Wenn der Gegenstand mit etwas Blickdichtem abgedeckt wird, wird das Licht blockiert. Der Zauber endet vorzeitig, wenn du ihn erneut wirkst oder ihn als Aktion aufhebst.]{} &
		
		\dndspell[Reparo][Zaubertrick der Verwandlung][Eine Minute][Berührung][V, S, M (Zauberstab)][Unmittelbar][Dieser Zauber repariert einen Bruch oder Riss eines von dir berührten Gegenstands, wie z.B. einer gebrochenen Kette, eines halbierten Schlüssels, oder eines zerrissenen Umhangs. Sofern der Bruch oder Riss nicht größer als 30cm in jede Richtung ist, reparierst du ihn ohne nachweisliche Spuren des Schadens zurückzulassen.][Dieser Zauber kann zwar einen magischen Gegenstand oder ein Konstrukt physisch reparieren, jedoch keine – dem Gegenstand innewohnende - Magie selbst wiederherstellen.]{} &
		
		\dndspell[Incendio][Zaubertrick der Hervorrufung][Eine Aktion][9 Meter][V, S, M (Zauberstab)][Unmittelbar][Ein flackerndes Feuer schießt aus deinem Zauberstab auf eine Kreatur in Reichweite zu. Führe einen Fernkampf-Zauberangriff gegen das Ziel aus. Bei einem Treffer erleidet es 1W8 Feuerschaden.][Der Schaden erhöht sich jeweils um 1W8 beim Erreichen des 5. Grades (2W8), 11. Grades (3W8), und 17. Grades (4W8).]{}
	\end{tabularx}

	\begin{tabularx}{\linewidth}{XXX}
		\dndspell[Episkey][Hervorrufung 1. Grades][Eine Aktion][Berührung][V, S, M (Zauberstab)][Unmittelbar][Eine Kreatur, die du berührst, gewinnt Trefferpunkte in Höhe von 1W8 + dein Zauberwirken-Attributsmodifikator (Weisheit) zurück. Dieser Zauber wirkt nicht auf Untote oder Konstrukte.][{\bfseries Auf höheren Graden:} Wirkst du diesen Zauber, indem du einen Zauberplatz 2. Grades oder höher nutzt, steigt die Heilung für jeden Grad über dem 1. um 1W8.]{} &
		
		\dndspell[Expelliarmus][Verzauberung 1. Grades][Eine Aktion][18 Meter][V, S, M (Zauberstab)][Unmittelbar][Das Ziel macht einen Weisheitsrettungswurf. Ist es dabei nicht erfolgreich, so lässt es seinen Zauberstab fallen und fällt auf den Boden (liegend).][{\bfseries Auf höheren Graden:} Wirkst du diesen Zauber, indem du einen Zauberplatz 2. Grades oder höher nutzt, kann für jeden Grad über dem 1. Eine zusätzliche Kreatur betroffen sein. Die Kreaturen müssen sich innerhalb von 9 Metern voneinander befinden, wenn du sie als Ziel wählst.]{} &
		
		\dndspell[Sectumsempra][Nekromantie 1. Grades][Eine Aktion][18 Meter][V, S, M (Zauberstab)][Unmittelbar][Führe einen Fernkampf-Zauberangriff gegen eine Kreatur in Reichweite aus. Bei einem Treffer erleidet sie 3W10 nekromantischen Schaden.][{\bfseries Auf höheren Graden:} Wirkst du diesen Zauber, indem du einen Zauberplatz 2. Grades oder höher nutzt, erhöht sich der Schaden für jeden Grad über dem 1. um 1W10.]{}
	\end{tabularx}
\end{landscape}

\section{Monster}

	\dndtextbox{
		\dndmonsterheader[Nathair's Scherge][Mittelgroßer Humanoider (Mensch), rechtschaffen böse][12][9 (2W8)][9m]
		\dndmonsterattributes[13 (+1)][13 (+1)][12 (+1)][13 (+1)][11 (+0)][9 (-1)]
		\dndmonsterinfo[Täuschen +2, Religion +2][Passive Wahrnehmung 10][Gemeinsprache][$\frac{1}{8}$ (25 XP)]
		\dndmonsterfeature{Dunkle Hingabe}{Nathair's Scherge ist bei Rettungswürfen gegen Verzaubern und Einschüchtern im Vorteil.}
		\dndmonsterfeature{Zaubern}{Nathair's Scherge kann alle Zaubersprüche verwenden.}
		\dndmonsterfeature{Dolch}{\dndattack{Nahkampfwaffenangriff}{+3}{1,5m}{ein Ziel}{4 (1W6+1) Stichschaden}}
	}
	\newpage

\end{document}
